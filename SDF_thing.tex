%%%%%%%%%%%%%%%%%%%%%%%%%%%%%%%%%%%%%%%%%%%%%%%%%%%%%%%%%%%%%%%
%
% Welcome to Overleaf --- just edit your LaTeX on the left,
% and we'll compile it for you on the right. If you open the
% 'Share' menu, you can invite other users to edit at the same
% time. See www.overleaf.com/learn for more info. Enjoy!
%
%%%%%%%%%%%%%%%%%%%%%%%%%%%%%%%%%%%%%%%%%%%%%%%%%%%%%%%%%%%%%%%


% Inbuilt themes in beamer
\documentclass{beamer}

% Theme choice:
\usetheme{CambridgeUS}

% Title page details: 

\usepackage{polynom}
\usepackage{amssymb}
\usepackage{amsmath}

\usepackage{bm}
\usepackage[misc]{ifsym}
%\setlength{\columnseprule}{1pt}
%\def\columnseprulecolor{\color{blue}}

\usepackage{longtable}
\usepackage{enumitem}
\usepackage{mathtools}
\usepackage[applemac]{inputenc}
\usepackage{tikz}
\usepackage{listings}
    \usepackage{color}                                            %%
    \usepackage{array}                                            %%
    \usepackage{longtable}                                        %%
    \usepackage{calc}                                             %%
    \usepackage{multirow}                                         %%
    \usepackage{hhline}                                           %%
    \usepackage{ifthen}                                           %%
  
\usepackage{lscape}     
\usepackage{tfrupee}
\usepackage{parskip}

\pagecolor{white}
\DeclareMathOperator*{\Res}{Res}
\DeclareMathOperator*{\equals}{=}
\DeclareMathOperator*{\pipe}{|}

\hyphenation{op-tical net-works semi-conduc-tor}
\def\inputGnumericTable{}  
\graphicspath{{./images/}}


\begin{document}
\newcommand{\bfr}[2]{\section{#1} \begin{frame}{#1} #2 \end{frame}}

	\title{Project 2 : EMS}
		\author{ Abhay Shankar K \\ Archit V. Ganvir \\ B. Vishal Mathews}

	\begin{frame}
    		\titlepage 
	\end{frame}
	
	\begin{frame}
		\tableofcontents
	\end{frame}


	\providecommand{\brak}[1]{\ensuremath{\left(#1\right)}}
	\providecommand{\sbrak}[1]{\ensuremath{\left[#1\right]}}
	\providecommand{\cbrak}[1]{\ensuremath{\left\{#1\right\}}}
	\newcommand{\req}{\noindent \textbf{Required: }}
	\providecommand{\rpr}[2]{\ensuremath{P_{#1}\left(#2\right)}} %random variable notation
	\providecommand{\spr}[1]{\ensuremath{P\left(#1\right)}} %simple notation
	\providecommand{\cpr}[2]{\ensuremath{\spr{#1 \pipe #2}}} %conditional probability

	\bfr{Overview}{
	
		According to Wikipedia,
		\begin{itemize}
			\item Education management information systems (EMIS) aim to collect, integrate, process, maintain and disseminate data and information.
			\item It must support decision-making, policy-analysis and formulation, planning, monitoring and management at all levels of an education system. 
			\item It is a system of people, technology, models, methods, processes, procedures, rules, and regulations that function together.
			\item It provides education leaders, decision-makers and managers at all levels with a comprehensive, integrated set of relevant, reliable, unambiguous and timely data and information to support them in completion of their responsibilities.
		\end{itemize}
		}	
	
	\bfr{Salient points}{
	\begin{itemize}
		\item An EMS must model the interaction between a set of students (in the mathematical sense) and a set of teachers. 
		\item Each set must be able to exchange information in specific formats (e.g. Assignments, Recordings, Grading), on specific forums.
		\item It must also allow for mutability of the aforementioned sets, and must acquire data effectively for new-joinees.
		\item A username-password database is also necessary. 
		\item The scholastic structure, with Faculty and Students organised into multiple schools and multiple branches per school, is optional. It may simply complicate matters.
		\item Courses are classified by field, subfield, then complexity. Each Course has member Sessions, Assignments, and Tests. 
	\end{itemize}
	}
	
	
	
	\bfr{Use Cases}{
	
		\begin{block}{Faculty}
		\begin{itemize}
		\item Uploads video session.
		\item Uploads assignment question.
		\item Uploads test question paper and answer paper (separately).
		\item Evaluate both Tests and Assignments.
		\end{itemize}
		\end{block}
		
		\begin{block}{Student}
		\begin{itemize}
		\item Watch video session.
		\item Uploads answer to assignment.
		\item Take tests.
		\end{itemize}
		\end{block}
	}
	
	\bfr{Forums}{
	
	For this project, a forum is a platform that facilitates a specific type of interaction between Faculty and Student. Each corresponding pair of Use-cases will also correspond to a specific forum.
	
	Thus, there will be a single forum for the submission and evaluation of assignments, and a separate one for taking tests.
		}
		
	
	
	
	
	
	
\end{document}